\section{\pivot Clause Semantics}
\label{sec:pivot}
The \pivot clause inputs a bag of binding tuples or an array of binding tuples.
Semantically, it is similar to \select \values but whereas the latter creates a
collection of values, \pivot constructs a tuple where the each input binding is
evaluated to an attribute value pair in the tuple.

The clause:

\begin{lstlisting}
PIVOT (*$v$*) AT (*$a$*)
\end{lstlisting}

\noindent inputs a bag or an array of binding tuples and outputs a single tuple
where each evaluation of $v$ and $a$ generate an attribute in the tuple.

\begin{example}
This example illustrates a \pivot that creates a tuple from a collection
of tuples.

\begin{lstlisting}
PIVOT x.v AT x.a
FROM << {'a': 'first', 'v': 'john'}, {'a': 'last', 'v': 'doe'} >>
\end{lstlisting}

\noindent The result is \lstinline|{'first':'john', 'last':'doe'}|.
\end{example}

The expression $a$ is expected to evaluate into a string value.  In strict mode,
it is an error if this evaluates to a non-string value.  In permissive mode, the
attribute is considered \MISSING and does not appear in the output.  The
expression $v$ can be any PartiQL value, but if it is \MISSING it will not be
generated in the resulting tuple.