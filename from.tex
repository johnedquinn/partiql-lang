\section{\from Clause Semantics}
\label{sec:from}

The formal semantics of a \from clause describe the collection of binding
tuples $B^{out}_{\from}$ that is output by the \from clause. The semantics
specify three cases and essentially extend the tuple calculus that underlies the
SQL semantics.

\begin{enumerate}
\item The semantics specify what is the core semantics of a \from clause with a
single \from item (Sections~\ref{sec:single-item-from} and ~\ref{sec:unpivot}).
The term ``semantics of the \from item $f$'' is synonymous to the term
``semantics of a \from clause with the single item $f$''. In either case, we
refer to the specification of the collection of binding tuples $B^{out}_{\from}$
that results from the evaluation of ``\from $f$''.

\item Then the semantics specify how multiple \from items combine, according to
the core semantics, using the join and outerjoin operations
(Sections~\ref{sec:combining-multiple-item-join},
\ref{sec:combining-multiple-item-leftjoin} and
~\ref{sec:combining-multiple-item-full-outerjoin}). 

\item Finally, the semantics specify the syntactic sugar structures that are
overlaid over the core semantics. Their primary purpose is SQL compatibility.
\end{enumerate}


\subsection{Ranging Over Bags and Arrays}
\label{sec:single-item-from}

Next we define the semantics of a \from clause that has a single \from item
and such item ranges over a bag or array. First consider the \from clause:

\begin{lstlisting}
FROM (*$a$*) AS (*$v$*) AT (*$p$*)
\end{lstlisting}

\noindent Let us call $v$ to be the \emph{element variable} and $p$ to be the
\emph{position variable}. In the normal case, $a$ is an array $[ e_0, \ldots,
e_{n-1} ]$. The \from clause outputs a bag of binding tuples. For each $e_i$,
the bag has a binding tuple $\langle v: e_i, p:i \rangle$.

\begin{example}
\label{xmpl:single-from-item-with-order}

Consider the following $\db$ (database environment):

\begin{lstlisting}
(*$\db = \langle$*)
    someOrderedTable:[
        {'a':0, 'b':0},
        {'a':1, 'b':1}
    ]
(*$\rangle$*)
\end{lstlisting}

\noindent then the following \from clause:

\begin{lstlisting}
FROM someOrderedTable AS x AT y
\end{lstlisting}

\noindent outputs the bag of binding tuples:

\begin{tabbing}
\ \ \ \=$B^{out}_{\from} = \ob $\=
    $\langle$ \lstinline|x:{'a':0, 'b':0}, y:0| $\rangle$\\
\>\>$\langle$ \lstinline|x:{'a':1, 'b':1}, y:1| $\rangle$\\
\>\>$\cb$
\end{tabbing}
\end{example}

As in SQL, the keyword \as is optional. The same applies to all cases below
where \as appears. If there is no \at clause, then the binding tuples have only
the element variable. In particular, consider:

\begin{lstlisting}
FROM (*$a$*) AS (*$v$*)
\end{lstlisting}

\noindent Normally $a$ is a collection, i.e, an array
$[ e_0, \ldots, e_{n-1} ]$ or a bag $\ob e_0, \ldots, e_{n-1} \cb$.
In either case, the \from clause outputs a bag. For each $e_i$, the bag
has a binding tuple $\langle v:e_i \rangle$.

\begin{example}
Consider again the database of Example~\ref{xmpl:single-from-item-with-order}
and then the \from clause

\begin{lstlisting}
FROM someOrderedTable AS x
\end{lstlisting}

\noindent this \from clause outputs:

\begin{tabbing}
\ \ \ \=$B^{out}_{\from} = \ob $\=
    $\langle$ \lstinline|x:{'a':0, 'b':0}| $\rangle$\\
\>\>$\langle$ \lstinline|x:{'a':1, 'b':1}| $\rangle$\\
\>\>$\cb$
\end{tabbing}
\end{example}

\subsubsection{Mistyping Cases}
\label{sec:bag-array-mistypings}

In the following cases the expression in the \from clause item has the wrong
type. Under the type checking option, all of these cases raise an error and the
query fails. Under the permissive option, the cases proceed as follows

\begin{itemize}
\item \highlight{Position variable on bags} Consider the clause:

\begin{lstlisting}
FROM (*$b$*) AS (*$v$*) AT (*$p$*)
\end{lstlisting}

\noindent and assume that $b$ is a bag $\ob e_0, \ldots, e_{n-1} \cb$. The
output is a bag with binding tuples $\langle v: e_i, p: \MISSING \rangle$. The
value \MISSING for the variable $p$ indicates that the order of elements in
the bag was meaningless. 

\item \highlight{Iteration over a scalar value} Consider the query:

\begin{lstlisting}
FROM (*$s$*) AS (*$v$*) AT (*$p$*)
\end{lstlisting}

\noindent or the query:

\begin{lstlisting}
FROM (*$s$*) AS (*$v$*)
\end{lstlisting}
 
\noindent where $s$ is a scalar value. Then $s$ coerces into the bag $\ob s
\cb$, i.e., the bag that has a single element, the $s$. The rest of the
semantics is identical to what happens when the lhs of the \from item is a bag.

\begin{example}
Consider again the database of Example~\ref{xmpl:single-from-item-with-order}
and the \from clause: 

\begin{lstlisting}
FROM someOrderedTable[0].a AS x
\end{lstlisting}

The expression \gl{someOrderedTable[0].a} evaluates to \gt{0} and,
consequently, the \from clause outputs a single binding tuple:

\begin{tabbing}
\ \ \ \=$B^{out}_{\from} = \ob
        \langle$ \lstinline|x:0| $\rangle \cb$\\
\end{tabbing}
\end{example}

\item \highlight{Iteration over a tuple value} Consider the query:

\begin{lstlisting}
FROM (*$t$*) AS (*$v$*) AT (*$p$*)
\end{lstlisting}

\noindent or the query:

\begin{lstlisting}
FROM (*$t$*) AS (*$v$*)
\end{lstlisting}
 
\noindent where $t$ is a tuple. Then $t$ coerces into the bag $\ob t \cb$

\item \highlight{Iteration over an absent value} Consider the query

\begin{lstlisting}
FROM (*$a$*) AS (*$v$*) AT (*$p$*)
\end{lstlisting}

\noindent or the query

\begin{lstlisting}
FROM (*$a$*) AS (*$v$*)
\end{lstlisting}

\noindent whereas $a$ evaluates into an \gn{absent\_value} (i.e., either
\MISSING or \NULL). In either case the \gn{absent\_value} $a$ coerces
into the bag $\ob a \cb$. Then the semantics follow the normal case.

\begin{example}
Consider again the database of Example~\ref{xmpl:single-from-item-with-order}
and the \from clause 

\begin{lstlisting}
FROM someOrderedTable[0].c AS x
\end{lstlisting}

The expression \gl{someOrderedTable[0].c} evaluates to \MISSING and,
consequently, the \from clause outputs the binding tuple:

\begin{tabbing}
\ \ \ \=$B^{out}_{\from} = \ob \langle$ \lstinline|x:MISSING| $\rangle \cb$\\
\end{tabbing}
\end{example}

\end{itemize}

\subsection{Ranging over Attribute-Value Pairs}
\label{sec:unpivot}

The \gl{UNPIVOT} clause enables ranging over the attribute-value pairs of a
tuple. The \from clause

\begin{lstlisting}
FROM UNPIVOT (*$t$*) AS (*$v$*) AT (*$a$*)
\end{lstlisting}

\noindent normally expects $t$ to be a tuple, with attribute/value pairs
$a1: v1, \ldots, a_n:v_n$. It does not matter whether the tuple is ordered
or unordered. The \from clause outputs the collection of binding tuples 

\[
    B^{out}_{\from} = \ob
        \langle v:v_1, a:a_1 \rangle 
        \ldots
        \langle v:v_n, a:a_n\rangle
    \cb 
\]

\begin{example}
Consider the $\db$:

\begin{lstlisting}
(*$\db = \langle$*)
    justATuple: {'amzn': 840.05, 'tdc': 31.06}
(*$\rangle$*)
\end{lstlisting}

\noindent The \from clause:

\begin{lstlisting}
FROM UNPIVOT justATuple AS price AT symbol
\end{lstlisting}

\noindent outputs:

\begin{tabbing}
\ \ \ \=$B^{out}_{\from} = \ob $\=
    $\langle $\lstinline|price: 840.05, symbol:'amzn'|$ \rangle$\\
\>\>$\langle $\lstinline|price: 31.06, symbol:'tdc'|$ \rangle$\\
\>\>$\cb$
\end{tabbing}
\end{example}

\subsubsection{Mistyping Cases}
\label{sec:unpivot-mistypings}

In the following cases the expression in the \from \unpivot clause item has the
``wrong" type, i.e., it is not a tuple. Under the type checking option, all of these cases raise an error
and the query fails. Under the permissive option, the cases proceed as follows:

\begin{lstlisting}
FROM UNPIVOT (*$x$*) AS (*$v$*) AT (*$n$*)
\end{lstlisting}

\noindent whereas $x$ is not a tuple and is not \MISSING, is equivalent to:

\begin{lstlisting}
FROM UNPIVOT {'_1': (*$x$*)} AS (*$v$*) AT (*$n$*)
\end{lstlisting}

\noindent Effectively, a tuple is generated for the non-tuple value.  When $x$ is \MISSING
then the above is equivalent to:

\begin{lstlisting}
FROM UNPIVOT {} AS (*$v$*) AT (*$n$*)
\end{lstlisting}

\noindent remember that a tuple cannot contain \MISSING. So the present case is equivalent to the empty tuple case.

\subsection{Combining Multiple \from Items with Comma, \CROSSJOIN, or \JOIN}
\label{sec:combining-multiple-item-join}

The \from clause expressions:

\begin{lstlisting}
(*$l$*) , (*$r$*) (*$\eqv$*)
(*$l$*) CROSS JOIN (*$r$*) (*$\eqv$*)
(*$l$*) JOIN (*$r$*) ON TRUE (*$\eqv$*)
\end{lstlisting}

\noindent have the same semantics. They combine the bag of bindings produced
from the \from item $l$ with the bag of binding tuples produced by the \from
item $r$, whereas the expression $r$ may utilize variables defined by $l$. Again, the term ``the semantics of $l\ \CROSSJOIN\ r$'' is equivalent
to the term ``the semantics of \from $l\ \CROSSJOIN\ r$''. In both cases, the
semantics specify a bag of binding tuples.

\paragraph{Associativity of $\CROSSJOIN$} We explain the $\CROSSJOIN$ and
``$,$'' as if they are left associative binary operators, despite the fact that
one can write more than two \from items without specifying grouping with
parenthesis. Since the ``$,$'' and $\CROSSJOIN$ operators are associative, we
may write (as is common in SQL):

\begin{lstlisting}
(*$f_1$*), (*$f_2$*), (*$f_3$*) (*$\eqv$*)
(*$f_1$*) CROSS JOIN (*$f_2$*) CROSS JOIN (*$f_3$*) (*$\eqv$*)
(*$f_1$*) LEFT JOIN (*$f_2$*) ON TRUE CROSS JOIN (*$f_3$*) ON TRUE (*$\eqv$*)
((*$f_1$*), (*$f_2$*)), (*$f_3$*) (*$\eqv$*)
((*$f_1$*) CROSS JOIN (*$f_2$*)) CROSS JOIN (*$f_3$*) (*$\eqv$*)
((*$f_1$*) LEFT JOIN (*$f_2$*) ON TRUE) CROSS JOIN (*$f_3$*) ON TRUE (*$\eqv$*)
\end{lstlisting}

\paragraph{Semantics} Consider the following:

\begin{lstlisting}
(*$l$*) CROSS JOIN (*r*)
\end{lstlisting}

\noindent unlike SQL, the rhs $r$ of the expression may use variables defined by
the lhs item $l$.  The result of this expression for a database environment
$\db$ and variables environment $\env$ is the bag of binding tuples produced by
the following pseudo-code. The pseudo-code uses the function $\evalf(\db, \env,
e)$ that evaluates the expression $e$ within the environments $\db$ and $\env$,
i.e., $\db, \env \vdash e \rightarrow \evalf(\db, \env, e)$.

\begin{tabbing}
\ \ \ \=for \=each binding tuple $b^l$ in $\evalf(\db, \env, l)$\\
\>\>for \=each binding $b^r$ in $\evalf(\db, (\env \| b^l), r)$\\
\>\>\>add $b^l \| b^r$ to the output bag
\end{tabbing}

In other words, the ``$l\ \CROSSJOIN\ r$'' outputs all binding tuples $b = b^l
\| b^r$, where $b^l \in \evalf(\db, \env, l)$ and $b^r \in \evalf(\db, (\env \|
b^l), r)$. The key extension to SQL is that $r$ is evaluated in the variables
environment $\env \| b^l$, i.e., it can use the variables that were defined by
$l$. The details of the variable scoping aspects are described in
Section~\ref{sec:scoping-variables}.

\begin{example}
This example simply reminds the tuple calculus explanation of the
\from SQL semantics. It does not yet endeavor into special aspects
of PartiQL. Consider the following database, which is conventional SQL:

\begin{lstlisting}
(*$\db = \langle$*)
    customers: [
        {'id': 5, 'name': 'Joe'},
        {'id': 7, 'name': 'Mary'}
    ],
    orders: [
        {'custId': 7, 'productId': 101},
        {'custId': 7, 'productId': 523}
    ]
(*$\rangle$*)
\end{lstlisting}

\noindent Then consider the following \from clause, which could be coming from
a conventional SQL query:

\begin{lstlisting}
FROM customers AS c, orders AS o
\end{lstlisting}

\noindent Note that in PartiQL this could also be written using the \CROSSJOIN
keyword, and presumably, one would put the sensible equality condition
\gt{c.id=o.custId} in the \gl{WHERE} clause. At any rate, this \from clause
outputs the bag of binding tuples:

\begin{tabbing}
\ \ \ \=$B^{out}_{\from} = \ob $\=
    $\langle$ \lstinline|c: {'id': 5, 'name': 'Joe'}, o: {'custId': 7, 'productId': 101}| $\rangle$\\
\>\>$\langle$ \lstinline|c: {'id': 5, 'name': 'Joe'}, o: {'custId': 7, 'productId': 523}| $\rangle$\\
\>\>$\langle$ \lstinline|c: {'id': 7, 'name': 'Mary'}, o: {'custId': 7, 'productId': 101}| $\rangle$\\
\>\>$\langle$ \lstinline|c: {'id': 7, 'name': 'Mary'}, o: {'custId': 7, 'productId': 523}| $\rangle$\\
\>\>$\cb$
\end{tabbing}

\end{example}

\noindent Due to scoping rules that will be justified and elaborated in
Section~\ref{sec:variable-scoping}, when the rhs of a $\CROSSJOIN$ is a path or a
function that uses a variable named $n$, such variable must be referred as
$\gt{@}n$. 

\begin{example}
Consider the database:

\begin{lstlisting}
(*$\db = \langle$*)
    sensors: [
        {'readings': [{'v': 1.3}, {'v': 2}]},
        {'readings': [{'v': 0.7}, {'v': 0.8}, {'v': 0.9}]}
    ]
(*$\rangle$*)
\end{lstlisting}

\noindent Intuitively, the following \from clause unnests the tuples that are
nested within the \gt{readings}.

\begin{lstlisting}
FROM sensors AS s, s.readings AS r
\end{lstlisting}

\begin{tabbing}
\ \ \ $B^{out}_{\from} = \ob$\=
  $\langle$ \lstinline|s: {'readings': [{'v': 1.3}, {'v': 2}]}, r: {v:1.3}| $\rangle$,\\
\>$\langle$ \lstinline|s: {'readings': [{'v': 1.3}, {'v': 2}]}, r: {v:2}| $\rangle$,\\
\>$\langle$ \lstinline|s: {'readings': [{'v': 0.7}, {'v': 0.8}, {'v': 0.9}]}, r: {'v':0.7}| $\rangle$,\\
\>$\langle$ \lstinline|s: {'readings': [{'v': 0.7}, {'v': 0.8}, {'v': 0.9}]}, r: {'v':0.8}| $\rangle$,\\
\>$\langle$ \lstinline|s: {'readings': [{'v': 0.7}, {'v': 0.8}, {'v': 0.9}]}, r: {'v':0.9}| $\rangle$,\\
\>$\cb$
\end{tabbing}
\end{example}

\subsection{Combining Multiple \from Items with \LEFTJOIN}
\label{sec:combining-multiple-item-leftjoin}

The \from clause expression:

\begin{lstlisting}
(*$l$*) LEFT CROSS JOIN (*$r$*) (*$\eqv$*)
(*$l$*) LEFT JOIN (*$r$*) ON TRUE
\end{lstlisting}

\noindent replicates SQL's \gl{LEFT JOIN} functionality and, in addition, it
also works for the case where the lhs of $r$ uses variables defined from $l$.

\yannis{OS: Do you really mean to make the CROSS necessary in the absence of ON?}

\almann{Yes, there is a parser problem if we don't, this was raised by Redshift
a while ago.}

Let's assume that the variables defined by $r$ are $v^r_1, \ldots, v^r_n$. The
result of evaluating $l\ \LEFTCJOIN\ r$ in environments $\db$ and $\env$ is the
bag of binding tuples produced by the following pseudocode, which also uses the
$\evalf$ function (See Section~\ref{sec:combining-multiple-item-join}). 

\begin{tabbing}
\ \ \ \=for \=each binding $b^l$ in $\evalf(\db, \env, l)$\\
\>\>$B^r \leftarrow \evalf(\db, (\env \| b^l), r)$\\
\>\>if \=$B^r$ is the empty bag\\
\>\>\>add $b^l \| \langle v^r_1:\NULL \ldots v^r_n:\NULL \rangle$ to the output bag \\
\>\>else\\
\>\>\>for \=each binding $b^r$ in $B^r$\\
\>\>\>\>add $b^l \| b^r$ to the output bag
\end{tabbing}

\begin{example}
Consider the database:

\begin{lstlisting}
(*$\db = \langle$*)
    sensors: [
        {'readings': [{'v':1.3}, {'v':2}]}
        {'readings': [{'v':0.7}, {'v':0.8}, {'v':0.9}]},
        {'readings': []}
      ]
(*$\rangle$*)
\end{lstlisting}

\noindent Notice that the value of the last tuple's \gt{readings} attribute is the empty
array. The following \from clause unnests the tuples that are nested within
the \gt{readings} but will also keep around the tuple with the empty
\gt{readings}. (See the last binding tuple.)

\begin{lstlisting}
FROM sensors AS s LEFT CROSS JOIN s.readings AS r
\end{lstlisting}

\begin{tabbing}
\ \ \ $B^{out}_{\from} = \ob$\=
  $\langle$ \lstinline|s: {'readings': [{'v':1.3}, {'v':2}]}, r: {'v':1.3}| $\rangle$,\\
\>$\langle$ \lstinline|s: {'readings': [{'v':1.3}, {'v':2}]}, r: {'v':2}| $\rangle$,\\
\>$\langle$ \lstinline|s: {'readings': [{'v':0.7}, {'v':0.8}, {'v':0.9}]}, r: {'v':0.7}| $\rangle$,\\
\>$\langle$ \lstinline|s: {'readings': [{'v':0.7}, {'v':0.8}, {'v':0.9}]}, r: {'v':0.8}| $\rangle$,\\
\>$\langle$ \lstinline|s: {'readings': [{'v':0.7}, {'v':0.8}, {'v':0.9}]}, r: {'v':0.9}| $\rangle$,\\
\>$\langle$ \lstinline|s: {'readings': []}, r: NULL| $\rangle$,\\
\>$\cb$
\end{tabbing}
\end{example}

\subsection{Combining Multiple \from Items with \FULLJOIN}
\label{sec:combining-multiple-item-full-outerjoin}

The \from clause expression:

\begin{lstlisting}
(*$l$*) FULL JOIN (*$r$*) ON *$c$*
\end{lstlisting}

\noindent replicates SQL's \gl{FULL JOIN} functionality. It
assumes that (alike SQL) the lhs of $r$ does not use variables defined from $l$. 
Thus, we do not discuss further.

\subsection{Expanding \JOIN and \LEFTJOIN with \on}
\label{sec:rewriting-on}

In compliance to SQL, the \JOIN and \LEFTJOIN have an optional \on
clause. The semantics of \gl{ON} can be explained as syntactic sugar over the core
PartiQL. They can also be explained by a simple extension of the semantics of
Sections~\ref{sec:combining-multiple-item-join},
~\ref{sec:combining-multiple-item-leftjoin}, and
~\ref{sec:combining-multiple-item-full-outerjoin}. The semantics of:

\begin{lstlisting}
(*$l$*) JOIN (*$r$*) ON (*$c$*)
\end{lstlisting}

\noindent are the following modification of the pseudocode of
Section~\ref{sec:combining-multiple-item-join}. The modification is the
inclusion of the underlined line.

\begin{tabbing}
for \=each binding $b^l$ in $\evalf(\db, \env, l)$\\
\>for \=each binding $b^r$ in $\evalf(\db, (\env \| b^l), r)$\\
\>\>\underline{if $\evalf(\db, (\env \| b^l \| b^r), c)$ is true}\\
\>\>\ \ \ add $b^l \| b^r$ to the output bag
\end{tabbing}

\noindent The semantics of:

\begin{lstlisting}
(*$l$*) LEFT JOIN (*$r$*) ON (*$c$*)
\end{lstlisting}

\noindent are the following modification of the pseudocode of
Section~\ref{sec:combining-multiple-item-leftjoin}. In essence, the \LEFTJOIN
\on outputs a tuple padded with \NULL whenever there is no binding of $r$
that satisfies the condition $c$.

\begin{tabbing}
for \=each binding $b^l$ in $\evalf(\db, \env, l)$\\
\>$B^r \leftarrow \evalf(\db, (\env \| b^l), r)$\\
\>$Q^r \leftarrow \ob \cb$\\
\>for \=each binding $b^r$ in $B^r$\\
\>\>if \=$\evalf(\db, (\env \| b^l \| b^r), c)$ is true\\
\>\>\>add $b^r$ in $Q^r$\\
\>if \=$Q^r$ is the empty bag\\
\>\>add $b^l \| \langle v^r_1:\NULL \ldots v^r_n:\NULL\rangle$ to the output bag \\
\>else\\
\>\>for \=each binding $b^r$ in $Q^r$\\
\>\>\>add $b^l \| b^r$ to the output bag
\end{tabbing}

\subsection{SQL's \gl{LATERAL}}
\label{sec:lateral}

SQL 2003 used the \gl{LATERAL} keyword to correlate \from clause items. In
the interest of compatibility with SQL, PartiQL also allows the use of the
keyword \gl{LATERAL}, though it does not do anything more than the comma itself
would do. That is ``$l\ \gl{, LATERAL}\ r$'' is equivalent to ``$l\ \gl{,}\
r$''.

