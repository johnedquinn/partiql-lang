\section{Coercion of subqueries} 
\label{sec:subquery-coercion}

In PartiQL, as is the case with SQL as well, expressions may involve SFW
subqueries (\linequery{sfw subquery}). PartiQL SFW subqueries are enclosed in
parentheses (i.e., identical to SQL). For compatibility with SQL, a SFW subquery
starting with a \gt{SELECT} clause (as opposed to a subquery starting with
\gt{SELECT VALUE} or \gl{PIVOT}) coerces into a scalar or into an array,
depending on the context. The following cases replicate SQL's coercing behavior
and analyze in which cases the result of a subquery coerces into scalar and in
which cases they coerce into arrays.

An PartiQL extension with respect to SQL is that, in the permissive mode,
subqueries that fail to coerce to the required type (scalar or tuple) still run,
as opposed to failing. They simply omit from the results the data that
correspond to the coercion failures.

\subsection{Coercion of a \gl{SELECT} subquery into a scalar}
\label{sec:select-coercion-scalar}
In each of the following cases a SFW subquery coerces into a scalar
\begin{itemize}
\item if it appears as the rhs of a comparison operator (\gt{=}, \gt{>}, etc)
where the lhs is not an array literal. And, vice versa, if it appears as the lhs
of a comparison operator where the rhs is not an array literal. (If it is the
lhs of a comparison operator where the lhs is an array literal, it coerces into
array, per Section~\ref{sec:select-coercion-array}.)
\item if it is an SFW subquery expression that (a) is not the collection
expression of a \gl{FROM} clause item and (b) is not the rhs of an \gl{IN}. (If
it is the rhs of an \gl{IN} then it should not be coerced.)
\eat{FIXME ; see note on semantics of \gl{IN}, Section~\ref{sec:in}.)}
\end{itemize}

Essentially, a subquery that is coerced may appear in all clauses except the
\gl{FROM}. For example, it may be a \gt{SELECT} subquery $s$ that appears as an
item of a \gl{SELECT}, \gl{SELECT VALUE} or \gl{PIVOT} clause. Or it may
be a subexpression of an expression that appears in \gl{SELECT}, \gl{SELECT
VALUE} or \gl{PIVOT} clause. Or it may be a subexpression of the
\gl{WHERE} clause expression, as long as it is not the rhs of an \gl{IN}. In any
of these cases the result of the subquery $s$ is cast into a scalar. 

Technically, the subquery $s$ (which uses \gl{SELECT}) is rewritten into an
equivalent subquery $s'$ that utilizes \gl{SELECT VALUE}, by following the steps
of Section~\ref{sec:sql-select}. Then the result of $s'$ is cast into a scalar
by applying the function \gl{COLL\_TO\_SCALAR($s'$)}. 

\begin{example} The SQL query
\begin{tabbing}
\ \ \ \=\gt{SELECT }\=\gt{v.foo,}\\ 
\>\>\gt{(SELECT w.bar}\\
\>\>\gt{\ FROM someDataSet w}\\
\>\>\gt{\ WHERE w.sth = v.sthelse) AS bar}\\
\>\gt{FROM anotherDataSet v}
\end{tabbing}
\noindent is rewritten into 
\begin{tabbing}
\ \ \ \=\gt{SELECT VALUE \{}\=\gt{'foo': v.foo}\\
\>\>\gt{'bar': COLL\_TO\_SCALAR(}\=\gt{SELECT VALUE \{'bar': w.bar\}}\\
\>\>\>\gt{FROM someDataSet w}\\
\>\>\>\gt{WHERE w.sth = v.sthelse)\}}\\
\>\gt{FROM anotherDataSet v}
\end{tabbing}
\end{example}

As is the common semantics of PartiQL in the permissive mode, when
\gl{COLL\_TO\_SCALAR} fails to cast the subquery into a scalar, it outputs
\MISSING. The inputs that are coerced into scalars are the ones that SQL
prescribes: When the input is a collection consisting of a single tuple with a
single attribute, the input is coerced into a scalar. All other inputs to
\gl{COLL\_TO\_SCALAR} lead to \MISSING.

\begin{example} 
\label{xmpl:coercion-failure}
In this example, in one instance the inner \gt{SELECT} evaluates to a collection
with more than one element. Because the \gl{COLL\_TO\_SCALAR} function produces a
\MISSING instead of failing, the query works. 

Consider the tables
\begin{tabbing}
\ \ \ \=\gt{customers : [}\=\gt{\{'id':1, 'name':'Mary'\},}\\
\>\>\gt{\{'id':2, 'name':'Helen'\},}\\
\>\>\gt{\{'id':1, 'name':'John'\}}\\
\>\>\gt{]}\\
\>\gt{orders : [}\=\gt{\{'custId':1, 'name':'foo'\},}\\
\>\>\gt{\{'custId':2, 'name':'bar'\}}\\
\>\>\gt{]}
\end{tabbing}

The following query would fail in SQL, because there are two customer tuples
with the same id. Of course, in a well-designed SQL database that has a primary
key or uniqueness constraint on the id, there would not be two customers with
the same id. However, lack of constraints is typical in the data targeted by
PartiQL. This query runs in the permissive mode of PartiQL.

\begin{tabbing}
\ \ \ \=\gt{SELECT }\=\gt{o.name AS orderName, }\\
\>\>\gt{(SELECT c.name FROM customers c WHERE c.id=o.custId) AS customerName}\\ 
\>\gt{FROM orders o}
\end{tabbing}
The result is
\begin{tabbing}
\gt{\ob\ \{'orderName':'foo'\}, \{'orderName':'bar', 'customerName':'Helen'\}\ \cb}
\end{tabbing}
\noindent Notice the missing \gt{'customerName'} in the first tuple.
\end{example} 

As in SQL, an implementation with static type checks will be able to detect and
warn that, in certain cases, a coercion will always fail and produce
\gt{missing}. 

\begin{example}
The following \gl{SELECT} clause is guaranteed to produce tuples with \gt{bar1}
and \gt{bar2}. Thus it cannot coerce into scalar. 
\begin{tabbing}
\ \ \ \=\gt{SELECT w.bar1 AS bar1, w.bar2 AS bar2}\\
\>\gt{FROM someDataSet w}
\end{tabbing}
Static type analysis can infer that the nested query above will deliver tuples
consisting of \gt{bar1} and \gt{bar2}. Thus, even before accessing any data, it
can warn the user that this query is erroneous.
\end{example}

\subsection{Coercion of a \gt{SELECT} subquery into an array}
\label{sec:select-coercion-array}

An \gt{SELECT} SFW subquery coerces into an array when it is the rhs
(respectively, lhs) of a comparison operator whose other argument is an array.

\footnote{Recall, in the interest of compatibility to SQL, PartiQL allows array
literals to be denoted with parentheses instead of brackets (see
Section~\ref{sec:array-constructor}).}

\eat{
Technically, a function \gl{COLL\_TO\_ARRAY} coerces the singleton collection of
ordered tuples produced by the \gl{SELECT} subquery into an array that has the
values of the single tuple of the collection, in the same order that they
appeared in the tuple. If the subquery outputs anything but a singleton
collection of ordered tuples, \gl{COLL\_TO\_ARRAY} returns a \MISSING  in the
permissive mode and fails in the type-checking mode.
}

The reduction of a \gl{SELECT} subquery to the PartiQL is exhibited by the
following example.
 
\begin{example} The SQL query
\begin{tabbing}
\ \ \ \=\gt{SELECT }\=\gt{v.foo}\\ 
\>\gt{FROM anotherDataSet v}\\
\>\gt{WHERE (v.a, v.b) = (}\=\gt{SELECT w.c, w,d}\\
\>\>\gt{FROM someDataSet w}\\
\>\>\gt{WHERE w.sth = v.sthelse)}
\end{tabbing}
\noindent is rewritten into 
\begin{tabbing}
\ \ \ \=\gt{SELECT VALUE \{'foo': v.foo\}}\\
\>\gt{FROM anotherDataSet v}\\
\>\gt{WHERE (v.a, v.b) = (}\=\gt{SELECT VALUE [w.c, w,d]}\\
\>\>\gt{FROM someDataSet w}\\
\>\>\gt{WHERE w.sth = v.sthelse)}
\end{tabbing}
\end{example}

\eat{
\[ \gt{SELECT}\ r \mapsto \gt{collToCollArrays(SELECT\ VALUES}\ r\gt{)} \]

The function \gt{collToCollArrays} coerces the collection produced by \gt{SELECT
VALUES} into a collection of arrays. In particular, if an element is an ordered
tuple, the tuple is coerced into an array by losing the attribute names. If an
element is anything but an ordered tuple or an array, it is dismissed by the
coercion, i.e., it does not appear in the result of the coercion.

 Scalar lhs: $l$\ \gt{IN (SELECT}\ $r$\gt{)}, when $l$ is an expression. 

When the lhs is an expression, the nested \gt{SELECT} is transformed into a
\gt{SELECT VALUE} that produces a collection of scalars, as it would be in SQL.
Technically

\[\gt{SELECT}\ s\ (\gt{AS}\ a) \]

is transformed into 

\[ \gt{SELECT VALUE }s \]

\reminder{Open Issue: Transformation in the presence of *, though it is almost
silly in this case.}

When the lhs is an array, the nested \gt{SELECT} in the rhs is transformed into
a \gt{SELECT VALUE} that produces a collection of arrays. In the case that the
\gt{SELECT} does not involve \gt{*} the transformation is based on treating the
\gt{SELECT} clause items as the elements of the array, potentially ignoring any
\gt{AS} clause. That is

 Array lhs: \gt{(}$l_1, l_2, \ldots$\gt{) IN (SELECT}\ $r$\gt{)}, where the lhs has two or more expressions $l_1, l_2, \ldots$. As in SQL, when there are two or more expressions the notation \gt{(}$l_1, l_2, \ldots$\gt{)} stands for an array whose $i$th element is the expression $l_i$. In contrast, \gt{)}$l$\gt{)} stands for a (scalar) expression.
}
