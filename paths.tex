\section{Path Navigation}
\label{section:paths}

\paragraph{Tuple path navigation} A \textit{tuple path navigation}
$t.a$ from the tuple $t$ to its attribute $a$ \linequery{tuple nav} returns the
value of the attribute $a$. (We discuss below the corner case where a tuple has
multiple attributes \texttt{a}.)  $t$ is an expression but $a$ is always an
identifier \linenames{identifier}. For example:

\begin{lstlisting}
{'a': 1, 'b':2}.a (*$\eqv$*) {'a': 1, 'b':2}."a" (*$\eval$*) 1    
\end{lstlisting}

\noindent Even if there were a variable \gt{a}, bound to \gt{'b'}, the result of
the above expression would still be \gt{1}, because the identifier \gt{a} (or
\gt{"a"}) is interpreted as the ``look for the attribute named \gt{a}'' when it
follows the dot in a tuple path navigation.  The semantics of tuple path
navigation do not depend on whether the tuple is ordered or unordered by schema.

\paragraph{Array navigation} An \textit{array navigation} $a[i]$ returns
the $i$-th element \textit{when} it is applied on an array $a$ \linequery{array
nav} and $i$ is an expression that evaluates into an integer. Both $a$ and $i$
are expressions. For example:

\begin{lstlisting}
[2, 4, 6][1+1] (*$\eval$*) 6. 
\end{lstlisting}

\paragraph{Tuple navigation with array notation} The expression $a[s]$ is
a shorthand for the tuple path navigation $a.s$ when the expression $s$ is
either (a) a string literal or (b) an expression that is explicitly \gl{CAST}
into a string. For example:

\begin{lstlisting}
{'a': 1, 'b': 2}['a'] (*$\eqv$*) {'a': 1, 'b':2}.'a' (*$\eval$*) 1
\end{lstlisting}

\noindent Similarly:

\begin{lstlisting}
{'attr': 1, 'b':2}[CAST('at' || 'tr' AS STRING)] (*$\eval$*) 1
\end{lstlisting}

If $s$ is not a string literal or an expression that is cast into a string, then
$a[s]$ is evaluated as an array path navigation. Notice that in the absence of
an explicit cast, the navigation $a[e]$ evaluates as an array navigation, even
if $e$ ends up evaluating to a string. For example, let us assume that the
variable \gt{v} is bound to \gt{'at'} and the variable \gt{w} is bound to
\gt{'tr'}. Still, the expression:

% TODO determine if cases where static type is known requires a CAST.

\begin{lstlisting}
{'attr': 1, 'b':2}[v || w]
\end{lstlisting}

\noindent does not evaluate to \gt{1}. It is treated as an array navigation with
wrongly typed index and it will return \gt{missing}, for reasons explained
below. 

\paragraph{Composition of navigations} Notice that consecutive
tuple/array navigations (e.g. \texttt{r.no[1]}) navigate deeply into complex
values. Notice further that paths consisting of plain tuple and array path
navigations evaluate to a unique value.

% TODO verify that this is correct for unordered tuples.

\paragraph{Tuple navigation in tuples with duplicate attributes} When the tuple
\gt{t} has multiple attributes \gt{a}, the tuple path navigation \gt{t.a} will
return the first instance of \gt{a}.  Note that for tuples whose order is
defined by schema, this is well-defined, for unordered tuples, it is
implementation defined which attribute is returned in \textit{permissive mode}
or an error in \textit{type checking mode}, which is described in
Section~\ref{sec:tuple-path-on-wrong}.

If one wants to access all instances of \gt{a}, she should use the \gt{UNPIVOT}
feature instead (see Section~\ref{sec:unpivot}). For example, the
following query returns the list of all \gt{a} values in a tuple \gt{t}.

\begin{lstlisting}
SELECT VALUE v
FROM UNPIVOT t AS v AT attr
WHERE attr = 'a'
\end{lstlisting}
 
\subsection{Tuple path evaluation on wrongly typed data}
\label{sec:tuple-path-on-wrong}

In the case of tuple paths, since PartiQL does not assume a schema, the
semantics must also specify the return value when:

\begin{compact_enum}
\item $t$ is not a tuple (i.e., when the expression $t$ does not evaluate into a
tuple), or

\item $t$ is a tuple that does not have an $a$ attribute.
\end{compact_enum}

\paragraph{Permissive mode} PartiQL can operate in a permissive mode or
in a conventional type checking mode, where the query fails once typing errors
(such as the above mentioned ones) happen. In the permissive mode, typing errors
are typically neglected by using the semantics outlined next.

In all of the above cases PartiQL returns the special value \gt{missing}.
Recall, the \gt{missing} is different from \gt{null}. The distinction enables
PartiQL to be able to distinguish between a tuple (JSON object) that lacked an
attribute \gt{a} and a tuple (JSON object) whose \gt{a} attribute was \gt{null}.
This distinction, coupled with appropriate features on how result tuples are
constructed (see \gt{SELECT} clause in Section~\ref{sec:select-values}), enables
PartiQL to easily preserve (when needed) the distinction between absent
attribute and null-valued attribute.

For example, the expression \gt{'not a tuple'.a} and the expression \gt{\{'a':1,
'b':2\}.noSuchAttribute} evaluate to \gt{missing}.

The above semantics apply regardless of whether the tuple navigation is
accomplished via the dot notation or via the array notation. For example, the
expression \gt{\{'a':1, 'b':2\}['noSuchAttribute']} will also evaluate to
\gt{missing}.

\paragraph{Type checking mode} In the type checking mode and in the
absence of schema, PartiQL will fail when tuple path navigation is applied on
wrongly typed data.

\subsubsection{Role of schema in type checking}
\label{sec:schema-in-tuple-path}

In the presence of schema, PartiQL may return a compile-time error when the
query processor can prove that the path expression is guaranteed to
\textit{always} produce \MISSING. The extent of error detection is
implementation-specific. 

For example, in the presence of schema validation, an PartiQL query processor
can throw a compile-time error when given the path expression \gt{\{a:1,
b:2\}.c}. In a more important and common case, an PartiQL implementation can
utilize the input data schema to prove that a path expression \textit{always}
returns \MISSING and thus throw a compile-time error. For example, assume
that \gt{sometable} is an SQL table whose schema does not include an attribute
\gt{c}. Then, an PartiQL implementation may throw a compile-time error when
evaluating the query:

\begin{lstlisting}
SELECT t.a, t.c FROM sometable AS t
\end{lstlisting}

Apparently, such an PartiQL implementation is fully compatible with the behavior
of an SQL processor. Generally, if a rigid schema is explicitly present, a tuple
path navigation error can be caught during compilation time; this is the case in
SQL itself, where referring to a non-existent attribute leads to a compilation
error for the query.

Notice that operating with schema validation may not prevent all tuple path
navigations from being applied to wrongly typed data. The choice between
permissive mode versus type checking mode dictates what happens next in these
cases: If permissive, the tuple path navigation evaluates into \MISSING. If
in type checking mode, the query fails.

\subsection{Array navigation evaluation on wrongly typed data}
\label{sec:array-on-wrong}

In the permissive mode, an array navigation evaluation $a[i]$ will result into
\gt{missing} in each of the following cases:

\begin{itemize}
\item $a$ does not evaluate into an array, or
\item $i$ does not evaluate into a positive integer within the array's bounds.
\end{itemize}

For example, \gt{[1,2,3][1.0]} evaluates to \gt{missing} since \gt{1.0} is not
an integer - even though it is coercible to an integer.
 
In type checking mode, the query will fail in each one of the cases above.

\subsection{Additional Path Syntax}
\label{sec:deep-navigation}

The following additional path functionalities are explained by reduction
to the basic tuple navigation and array navigation.

\paragraph{Wildcard steps} The expression $e[*]$ reduces to (i.e., is
equivalent to):

\begin{lstlisting}
SELECT VALUE (*$v$*) FROM (*$e$*) AS (*$v$*)
\end{lstlisting}

\noindent where $v$ is a \textit{fresh variable}, i.e., a variable that does not
already appear in the query.  Similarly, when the expression $e.*$ is not a
\gl{SELECT} clause item of the form $t.*$, where $t$ is a variable, it reduces
to:

\begin{lstlisting}
SELECT VALUE (*$v$*) FROM UNPIVOT (*$e$*) AS (*$v$*)
\end{lstlisting}

\noindent where $v$ is a fresh variable. An expression $t.*$, where $t$ is a
variable and the expression appears as a \gl{SELECT} clause item, is interpreted
according to the \gl{SELECT} clause \gl{*} semantics
(Section~\ref{sec:sql-star}).

\begin{example} 
The expression:

\begin{lstlisting}
[1,2,3][*] (*$\eqv$*) SELECT VALUE v FROM [1, 2, 3] AS v
    (*$\eval$*) <<1, 2, 3>>
\end{lstlisting}

\noindent The expression:

\begin{lstlisting}
{'a':1, 'b':2}.* (*$\eqv$*) SELECT VALUE v FROM UNPIVOT {'a':1, 'b':2} AS v
    (*$\eval$*) <<1, 2>>
\end{lstlisting}

\noindent Whereas the following query:

\begin{lstlisting}
SELECT t.* FROM <<{'a':1, 'b':1}, {'a':2, 'b':2}>> AS t
    (*$\eval$*) <<{'a':1, 'b':1}, {'a':2, 'b':2}>>
\end{lstlisting}

\noindent does not do the transformation with \gl{UNPIVOT}.  If one does not
want this behavior, \gl{SELECT VALUE} can be used
(Section~\ref{sec:select-values}).
\end{example}

\paragraph{Path Expressions with Wildcards}
PartiQL also provides multi-step path expressions, called \textit{path
collection expressions}. Their semantics is a generalization of the semantics of
a path expression with a single $[*]$ or $.*$. Consider the path collection
expression:

\[ e w_1 p_1 \ldots w_n p_n \]

\noindent where $e$ is any expression; $n>0$; each \textit{wildcard step}
$w_i$ is either $[*]$ or $.*$; each \textit{series of plain path steps} $p_i$ is
a sequence of zero or more tuple path navigations or array navigations
(potentially mixed). 

Then the path collection expression is equivalent to the SFW query 

\begin{lstlisting}
SELECT VALUE (*$v_n.p_n$*)
FROM
    (*$u_1$*) (*$e$*) AS (*$v_1$*),
    (*$u_2$*) @(*$v_1.p_1$*) AS (*$v_1$*),
    (*$\ldots$*),
    (*$u_n$*) @(*$v_{n-1}.p_{n-1}$*) AS (*$v_n$*),
\end{lstlisting}

\noindent where each $v_i$ is a fresh variable and each $u_i$ is \unpivot\ if
$w_i$ is a $.*$ and it is nothing if $w_i$ is a $[*]$. Intuitively $v_i$
corresponds to the $i$-th star.

\begin{example} According to the above, consider the following query:

\begin{lstlisting}
SELECT VALUE foo FROM e.* AS foo
\end{lstlisting}

\noindent reduces to

\begin{lstlisting}
SELECT VALUE foo FROM (SELECT VALUE v FROM UNPIVOT e AS v) AS foo
\end{lstlisting}

\noindent which is equivalent to

\begin{lstlisting}
SELECT VALUE foo FROM UNPIVOT e AS foo
\end{lstlisting}

\noindent Next, consider the path collection expression:

\begin{lstlisting}
tables.items[*].product.*.nest
\end{lstlisting}

\noindent This expression reduces to 

\begin{lstlisting}
SELECT
  VALUE v2.nest
FROM
  tables.items AS v1,
  UNPIVOT @v1.product AS v2
\end{lstlisting}

\end{example}
