\section{Introduction}
\label{sec:introduction}

\paragraph{Draft Status} This document is currently a working draft and subject
to change.  Certain sections are marked as ``work in progress'' (WIP) and will
be expanded soon.

\paragraph{Audience} This document presents the formal syntax and semantics of
PartiQL. It is oriented to PartiQL query processor builders who need the full
and formal detail on PartiQL.

SQL users who are not interested in the full detail and the complete formalism
but are interested in learning how PartiQL extends SQL may also read the
tutorial. Unlike this formal specification, the tutorial has a ``how to''
orientation and is primarily driven by examples. 

\paragraph{PartiQL core and PartiQL syntactic sugar}
In the interest of precision and succinctness, we tier the PartiQL specification
in two layers: The PartiQL core is a functional programming language with
composable aspects. Three aspects of the PartiQL core syntax and semantics are
characteristic of its functional orientation: Every (sub)query and every (sub)
expression input and output PartiQL data. Second, each clause of a SELECT query
is itself a function. Third, every (sub)query evaluates within the environment
created by the database names and the variables of the enclosing queries.

Then we layer ``syntactic sugar'' features over the core. Commonly, syntactic
sugar achieves well-known SQL syntax and semantics. Formally, every syntactic
sugar feature is explained by reduction to the core.

\pagebreak
