\section{\gl{UNION} / \gl{INTERSECT} / \gl{EXCEPT} clauses}
\label{section:setop}
Coming up...
\eat{
We describe next the PartiQL \texttt{UNION ALL}, \texttt{INTERSECT ALL} and
\texttt{EXCEPT ALL} bag operators, as well as their duplicate-eliminating
counterparts the \texttt{UNION}, \texttt{INTERSECT} and \texttt{EXCEPT}. While
SQL bag operators always input/output bags of flat, homogeneous tuples, PartiQL
queries have to face the following issues:
\begin{itemize}
\item the inputs may not be bags. Thus PartiQL semantics specify which cases
coerce and how Vs which cases fail.
\item the elements of the inputs may not be homogenous and they may not be
atomic values. Thus PartiQL semantics specify how comparisons between elements
are accomplished. 
\end{itemize}

\subsection{PartiQL Bag Operator Basics}
\label{sec:bagops-basics}
As in SQL, a bag operator%

\footnote{SQL's bag operators are often referred to as set operators, while the
term set is mathematically incorrect.}

is specified with: $q~S~q'$ (Figure \ref{figure:query:bnf}, lines 9-10), which
comprises a left query $q$, the bag operator $S$ and a right query $q'$. The bag
operator $S$ may be \texttt{UNION}, \texttt{INTERSECT} or \texttt{EXCEPT} and
may be optionally suffixed with \texttt{ALL}. Let $q \rightarrow v$ and $q'
\rightarrow v'$, where $v$ and $v'$ are arbitrary values. The \texttt{ALL}
variations input bags and output a bag of values, which may have duplicate
values even if the input does not. Presence of \texttt{ALL} specifies that the
output may have duplicate elements, while absence requires duplicate
elimination, in which case the output is an implicit set.

\subsection{Equivalence function used by Bag Operators}
\label{sec:equiv-bagops}
The bag operators utilize the function \gl{eqg} to decide the equivalence of
elements of their input. Notice it is the same function that is used by the
\gl{GROUP BY}, as described in Section~\ref{sec:eqg}. Furthermore, alike the
\gl{GROUP BY}, the \gl{UNION} and \gl{INTERSECT} operators retain the \gl{NULL}
whenever they compare a \gl{NULL} and a \gl{MISSING}.

\subsection{Operations on Non-Bags}
\label{sec:setops-on-non-bags}
Mistyping situations may emerge when one or both of the inputs to a set operator
is not a bag. In all cases, mistypings are treated identically to mistyping in
the \gl{FROM} clause, as described in Section~\ref{sec:bag-array-mistypings}. In
particular, if an input argument is any of the following, it is coerced
according to the following rules:

\begin{enumerate}
\item A scalar value $s$ is coerced into the bag $\ob s \cb$.
\item A tuple $t$ is coerced into the bag $\ob t \cb$.
\item An absent value $a$ is coerced into the bag $\ob a \cb$.
\item An array $r$ is coerced into the bag that dismisses the order of elements
in $r$.
\end{enumerate}

\subsection{Array versions of the Set Operators}
\label{sec:array-versions-of-setops}
SQL does not allow the set operators applied on ordered subqueries (i.e.,
queries with \gl{ORDER BY}). Thus there is no compatibility issue when we extend
the set operators of PartiQL to input arrays instead of bags. PartiQL allows its
users to easily express versions of the set operators where the inputs have
order and such order is easily preserved in the output.%

\footnote{The described functionality can also be accomplished without the array
operators that preserve input order. However, it is a tedious exercise to do the
job by using the \gl{AT} and \gl{ORDER BY} operators.}

\subsubsection{CONCAT} 
\label{sec:concat}
The query 
\[e_1\ \gl{CONCAT}\ e_2\] \noindent normally expects that $e_1$ and $e_2$ are
arrays. Its output is the concatenation of $e_1$ and $e_2$. Thus, it can be
thought of as the array version of \gl{UNION ALL}.

If $e_1$ is a bag and $e_2$ is an array, then the elements of $e_1$ appear
before the elements of $e_2$ in the output and the elements of $e_2$ maintain
their order. (Vice versa if $e_1$ is an array and $e_2$ is a bag.) If both $e_1$
and $e_2$ are bags, then the elements of $e_1$ appear before the elements of
$e_2$ in the output.

If $e_1$ or $e_2$ is neither an array nor a bag, it is conceptually coerced into
a singleton array (or singleton bag, since there is no difference) according to
the rules of Section~\ref{sec:setops-on-non-bags} and then proceed according to
the above semantics.

\subsubsection{ORDERED INTERSECT ALL}
\label{sec:ordered-intersect}
The query 
\[e_1\ \gl{ORDERED INTERSECT ALL}\ e_2\] \noindent normally expects that $e_1$
is an array and $e_2$ is either bag or array. It outputs the elements of $e_1$
that are found in $e_2$ and the qualified elements retain their order.

If $e_1$ is a bag the semantics are identical to \gl{INTERSECT ALL}.

If $e_1$ or $e_2$ is neither an array nor a bag, it is conceptually coerced into
a singleton array (or singleton bag, since there is no difference) according to
the rules of Section~\ref{sec:setops-on-non-bags} and then proceed according to
the above semantics.

\subsubsection{ORDERED EXCEPT ALL}
\label{sec:ordered-except}
The query 
\[e_1\ \gl{ORDERED EXCEPT ALL}\ e_2\] \noindent normally expects that $e_1$ is
an array and $e_2$ is either bag or array. It outputs the elements of $e_1$ that
are \textit{not} found in $e_2$ and the qualified elements retain their order.

If $e_1$ is a bag the semantics are identical to \gl{EXCEPT ALL}.

If $e_1$ or $e_2$ is neither an array nor a bag, it is conceptually coerced into
a singleton array (or singleton bag, since there is no difference) according to
the rules of Section~\ref{sec:setops-on-non-bags} and then proceed according to
the above semantics.

\yannis{to do: fix the grammar to (Figure~\ref{figure:query:bnf}.} 
}
